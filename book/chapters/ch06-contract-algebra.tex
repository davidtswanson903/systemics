\section{Chapter 6. Contract Algebra and Conformance Preservation (normative)}
\label{sec:partII-ch6}

\paragraph{Goal.} Make ``guarantees compose'' a theorem, not a slogan.

\subsubsection{Induced contract operators (normative)}

Let $\mathfrak{P}$ be a process algebra instance (Definition~\ref{def:process-algebra-instance}) with
a declared composition kit $\mathfrak{K}_{\mathrm{kit}}$ (Definition~\ref{def:composition-kit}).
Write $(\Compat_{\seq},\Compose_{\seq})$ for the sequential kit and (optionally)
$(\Compat_{\par},\Compose_{\par})$ for the parallel kit.

\begin{definition}[Sequential contract composition]
\label{def:cseq}
For contracts $\mathcal{C}_1,\mathcal{C}_2\subseteq T^\star$, define the \emph{sequential contract composition}
operator $\cseq$ by:
\[
\mathcal{C}_2 \cseq \mathcal{C}_1
\ :=\
\big\{\, \Compose_{\seq}(\tau_1,\tau_2)\ :\ \tau_1\in \mathcal{C}_1,\ \tau_2\in \mathcal{C}_2,\ \Compat_{\seq}(\tau_1,\tau_2)\,\big\},
\]
where $\Compose_{\seq}(\tau_1,\tau_2)$ is understood to be defined whenever $\Compat_{\seq}(\tau_1,\tau_2)$ holds
(as required by the constructor soundness schema, Axiom~\ref{ax:T0}).
\end{definition}

\begin{definition}[Parallel contract composition (optional)]
\label{def:cpar}
If $(\Compat_{\par},\Compose_{\par})$ is present, define the \emph{parallel contract composition}
operator $\cpar$ by:
\[
\mathcal{C}_1 \cpar \mathcal{C}_2
\ :=\
\big\{\, \Compose_{\par}(\tau_1,\tau_2)\ :\ \tau_1\in \mathcal{C}_1,\ \tau_2\in \mathcal{C}_2,\ \Compat_{\par}(\tau_1,\tau_2)\,\big\}.
\]
\end{definition}

\begin{remark}
The contract operators are \emph{induced} by the same kit as kernel composition. This is the central design rule:
\emph{the constructors that build composed traces are the constructors that build composed guarantees}.
\end{remark}

\subsubsection{Conformance preservation (normative)}

\begin{theorem}[Sequential conformance preservation]
\label{thm:seq-conformance-preservation}
Let $K_1,K_2\in \mathcal{K}$ and let $\mathcal{C}_1,\mathcal{C}_2\subseteq T^\star$ be contracts.
Assume:
\begin{enumerate}[leftmargin=*, itemsep=0.2em]
  \item $K_1 \models \mathcal{C}_1$,
  \item $K_2 \models \mathcal{C}_2$,
  \item (\emph{compositional generation}) every trace of $K_2\seqc K_1$ arises from composing
        a compatible pair of traces from $K_1$ and $K_2$ via the sequential constructor, i.e.:
        \[
        \forall \tau\in \Exec_{K_2\seqc K_1}\ \exists \tau_1\in \Exec_{K_1},\ \tau_2\in \Exec_{K_2}:
        \Compat_{\seq}(\tau_1,\tau_2)\ \wedge\ \tau=\Compose_{\seq}(\tau_1,\tau_2).
        \]
\end{enumerate}
Then:
\[
K_2\seqc K_1 \models (\mathcal{C}_2 \cseq \mathcal{C}_1).
\]
\end{theorem}

\begin{proof}
Let $\tau\in \Exec_{K_2\seqc K_1}$. By assumption (3), there exist $\tau_1\in\Exec_{K_1}$ and
$\tau_2\in\Exec_{K_2}$ such that $\Compat_{\seq}(\tau_1,\tau_2)$ and $\tau=\Compose_{\seq}(\tau_1,\tau_2)$.
From $K_1\models \mathcal{C}_1$ we have $\tau_1\in \mathcal{C}_1$, and from $K_2\models \mathcal{C}_2$ we have
$\tau_2\in \mathcal{C}_2$. Therefore, by Definition~\ref{def:cseq},
\[
\tau=\Compose_{\seq}(\tau_1,\tau_2)\in \mathcal{C}_2\cseq \mathcal{C}_1.
\]
Since $\tau$ was arbitrary, $\Exec_{K_2\seqc K_1}\subseteq (\mathcal{C}_2\cseq \mathcal{C}_1)$, i.e.\
$K_2\seqc K_1 \models (\mathcal{C}_2\cseq \mathcal{C}_1)$.
\end{proof}

\begin{theorem}[Parallel conformance preservation (optional)]
\label{thm:par-conformance-preservation}
If $\parc$ and $(\Compat_{\par},\Compose_{\par})$ are present and satisfy the analogous
compositional generation property for $K_1\parc K_2$, then:
\[
K_1\models \mathcal{C}_1\ \wedge\ K_2\models \mathcal{C}_2
\quad\Rightarrow\quad
K_1\parc K_2 \models (\mathcal{C}_1\cpar \mathcal{C}_2).
\]
\end{theorem}

\begin{proof}
Identical structure to Theorem~\ref{thm:seq-conformance-preservation}, replacing $\seqc,\Compat_{\seq},\Compose_{\seq},\cseq$
by $\parc,\Compat_{\par},\Compose_{\par},\cpar$.
\end{proof}

\subsubsection{Merge-closed contracts and closure corollaries (normative)}

\begin{definition}[Merge-closed contract (sequential)]
\label{def:merge-closed}
A contract $\mathcal{C}\subseteq T^\star$ is \emph{merge-closed for sequential composition} if
for all $\tau_1,\tau_2\in \mathcal{C}$,
\[
\Compat_{\seq}(\tau_1,\tau_2)\ \Rightarrow\ \Compose_{\seq}(\tau_1,\tau_2)\in \mathcal{C}.
\]
Equivalently, $\mathcal{C}\cseq \mathcal{C}\subseteq \mathcal{C}$.
\end{definition}

\begin{corollary}[Closure under sequential composition]
\label{cor:closure-seq}
If $\mathcal{C}$ is merge-closed (Definition~\ref{def:merge-closed}) and
$K_1\models \mathcal{C}$ and $K_2\models \mathcal{C}$, then:
\[
K_2\seqc K_1 \models \mathcal{C}.
\]
\end{corollary}

\begin{proof}
By Theorem~\ref{thm:seq-conformance-preservation}, we have
$K_2\seqc K_1 \models (\mathcal{C}\cseq \mathcal{C})$.
By merge-closure, $\mathcal{C}\cseq \mathcal{C}\subseteq \mathcal{C}$, hence
$K_2\seqc K_1 \models \mathcal{C}$.
\end{proof}

\begin{remark}
Merge-closed contracts are the mechanism by which ``global invariants'' arise: you state one contract $\mathcal{C}$
and prove it is closed under your composition kit. Then any pipeline built from conforming stages conforms globally.
\end{remark}
