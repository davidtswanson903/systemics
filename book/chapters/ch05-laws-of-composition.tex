\section{Chapter 5. Laws of Composition (normative)}
\label{sec:partII-ch5}

\paragraph{Goal.} The minimal equational basis for ``algebra of processes.''

\subsubsection{Equivalence and its status (normative)}

\begin{definition}[Kernel equivalence placeholder]
\label{def:eq-placeholder}
Fix a binary relation $\simeq\ \subseteq\ \mathcal{K}\times \mathcal{K}$ called \emph{kernel equivalence}.
In this chapter, $\simeq$ is treated as a \emph{primitive judgment} whose induction principle is deferred.
Later chapters either:
\begin{itemize}[leftmargin=*, itemsep=0.15em]
  \item \emph{derive} $\simeq$ from an observable semantics (e.g.\ trace views), or
  \item \emph{adopt} $\simeq$ axiomatically as part of the algebraic presentation.
\end{itemize}
\end{definition}

\begin{remark}
This is the standard algebraic move: we may state the equational theory at the process-algebra level,
and later justify it by semantics (soundness/completeness), or treat it as an axiomatic algebra
whose models are semantics structures satisfying the laws.
\end{remark}

\subsubsection{Congruence laws (normative)}

\begin{definition}[Congruence for an operator]
\label{def:congruence}
Let $\odot$ be a binary operator on $\mathcal{K}$. The relation $\simeq$ is a \emph{congruence} for $\odot$ if
\[
K_1\simeq K_1' \ \wedge\ K_2\simeq K_2' \quad \Rightarrow \quad (K_2\odot K_1)\simeq (K_2'\odot K_1').
\]
\end{definition}

\begin{axiom}[L1: Congruence for sequential composition]
\label{ax:L1-cong-seq}
$\simeq$ is a congruence for $\seqc$ (Definition~\ref{def:congruence}).
\end{axiom}

\begin{axiom}[L2: Congruence for parallel composition (optional)]
\label{ax:L2-cong-par}
If $\parc$ is present, then $\simeq$ is a congruence for $\parc$.
\end{axiom}

\subsubsection{Associativity and identities up to equivalence (normative)}

\begin{axiom}[L3: Associativity of $\seqc$ up to $\simeq$]
\label{ax:L3-assoc-seq}
For all kernels $K_1,K_2,K_3\in\mathcal{K}$,
\[
(K_3\seqc K_2)\seqc K_1 \ \simeq\ K_3\seqc (K_2\seqc K_1).
\]
\end{axiom}

\begin{axiom}[L4: Identity laws for $\seqc$ up to $\simeq$ (recommended)]
\label{ax:L4-id-seq}
If $\mathrm{Id}$ (or $\mathrm{Id}_A$) is present, then for all kernels $K$ (and all compatible types when typed):
\[
K\seqc \mathrm{Id}\ \simeq\ K,
\qquad
\mathrm{Id}\seqc K\ \simeq\ K.
\]
\end{axiom}

\begin{remark}
Identity laws are recommended because they enable normalization rules that eliminate no-op stages.
If identities are omitted, later chapters may still achieve a ``weak identity'' via normalization conventions,
but the algebra becomes less canonical.
\end{remark}

\subsubsection{Optional monoidal laws for $\parc$ (normative if adopted)}

If parallel composition is treated as a monoidal structure, record the laws explicitly.

\begin{axiom}[L5: Associativity of $\parc$ up to $\simeq$ (optional)]
\label{ax:L5-assoc-par}
If $\parc$ is present, then for all $K_1,K_2,K_3$,
\[
(K_1\parc K_2)\parc K_3 \ \simeq\ K_1\parc (K_2\parc K_3).
\]
\end{axiom}

\begin{axiom}[L6: Unit for $\parc$ up to $\simeq$ (optional)]
\label{ax:L6-unit-par}
If $\parc$ is present, there exists a distinguished unit kernel $\mathbf{1}$ such that for all $K$,
\[
K\parc \mathbf{1}\ \simeq\ K,
\qquad
\mathbf{1}\parc K\ \simeq\ K.
\]
\end{axiom}

\begin{axiom}[L7: Symmetry/commutativity for $\parc$ up to $\simeq$ (optional)]
\label{ax:L7-sym-par}
If $\parc$ is present and intended as symmetric, then for all $K_1,K_2$,
\[
K_1\parc K_2 \ \simeq\ K_2\parc K_1.
\]
\end{axiom}

\begin{remark}
Whether $\parc$ should be symmetric is instance-dependent. Many domains require ordered parallel composition
(e.g.\ when receipts record left/right provenance); in such cases omit L7 and later enforce a canonical ordering
via normalization rather than symmetry.
\end{remark}

\subsubsection{A compact axiom table (output) (normative)}

\begin{definition}[Composition law table]
\label{def:law-table}
A \emph{composition law table} for a process algebra instance $\mathfrak{P}$ (Definition~\ref{def:process-algebra-instance})
is a selection of the axioms:
\[
\textbf{Laws}(\mathfrak{P}) \subseteq \{\text{L1},\text{L2},\text{L3},\text{L4},\text{L5},\text{L6},\text{L7}\},
\]
where:
\begin{itemize}[leftmargin=*, itemsep=0.15em]
  \item L1 and L3 are the minimal sequential basis (congruence + associativity up to $\simeq$),
  \item L4 is recommended (identities up to $\simeq$),
  \item L2, L5--L7 are optional parallel/monoidal laws.
\end{itemize}
\end{definition}

\begin{remark}
Later chapters either \emph{prove} the selected laws from semantics (e.g.\ by defining $\simeq$ observationally
and showing congruence/associativity), or treat the law table as axioms and study the class of models (semantics)
that satisfy them.
\end{remark}
