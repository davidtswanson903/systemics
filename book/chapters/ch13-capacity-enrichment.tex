% book/chapters/ch13-capacity-enrichment.tex

\subsection{Chapter 13. Capacity Enrichment (normative)}
\label{sec:partV-ch13}

\paragraph{Goal.} Treat resources as first-class and compose feasibility.

\subsubsection{Capacity space and cost semantics (normative)}

\begin{definition}[Capacity space]
\label{def:cap-space}
A \emph{capacity space} is a poset $(C,\preceq)$ with least element $0_C$.
We interpret $c_1\preceq c_2$ as ``$c_2$ admits at least as much resource as $c_1$'' \emph{or}
``$c_1$ uses no more resource than $c_2$,'' depending on the adopted convention.
\end{definition}

\begin{remark}
This chapter adopts the \emph{usage} convention: smaller is better.
A capacity bound is written $C_{\max}\in C$, and an execution is feasible when $\Cost(\tau)\preceq C_{\max}$.
If you prefer the \emph{available budget} convention, reverse the inequalities systematically.
\end{remark}

\begin{definition}[Cost functional]
\label{def:cost}
A \emph{cost functional} is a map
\[
\Cost:\ T^\star \to C,
\]
assigning a capacity value (resource usage) to each trace.
\end{definition}

\begin{axiom}[Cost-field agreement (optional)]
\label{ax:cost-field}
If traces carry a distinguished capacity component $c(\tau)\in C$, then
\[
\Cost(\tau)=c(\tau)\qquad\text{for all }\tau\in T^\star.
\]
\end{axiom}

\subsubsection{Capacity-bounded execution (normative)}

\begin{definition}[Capacity-bounded trace set]
\label{def:exec-bounded}
For a kernel $K$ and bound $C_{\max}\in C$, define the capacity-bounded execution trace set:
\[
\Exec^{\le C_{\max}}_K \ :=\ \{\ \tau\in \Exec^\star_K\ :\ \Cost(\tau)\preceq C_{\max}\ \}.
\]
\end{definition}

\begin{definition}[Capacity-bounded conformance]
\label{def:cap-bounded-conf}
A kernel $K$ \emph{conforms to a contract $\mathcal{C}$ under bound $C_{\max}$} if
\[
\Exec^{\le C_{\max}}_K \subseteq \mathcal{C}.
\]
\end{definition}

\begin{remark}
This is the disciplined form of ``the guarantee holds when we stay within resource budget.''
\end{remark}

\subsubsection{Targets, achievement, and feasibility regions (normative)}

\begin{definition}[Target space]
\label{def:target-space}
Let $\mathcal{T}$ be a \emph{target space}. Minimally, $\mathcal{T}$ contains floor settings and (if Chapter~12 is adopted) stability budgets.
A canonical choice is:
\[
\mathcal{T} := \ThetaSpace \times \BetaSpace,
\]
where $\ThetaSpace$ is a poset of floors and $\BetaSpace$ is a poset of invariance budgets.
\end{definition}

\begin{definition}[Achievement predicate]
\label{def:achieve}
Fix $K$ and envelope $\gamma\in \Gamma$. An \emph{achievement predicate} is a map
\[
\Ach_{K,\gamma}:\ \mathcal{T}\times C \to 2,
\]
where $\Ach_{K,\gamma}(t,C_{\max})=1$ means:
``under capacity bound $C_{\max}$, the kernel achieves target $t$ in envelope $\gamma$.''

Achievement is defined by a declared contract family; e.g.\ for $t=(\Theta,\beta)$:
\[
\Ach_{K,\gamma}((\Theta,\beta),C_{\max})=1
\quad\Longleftrightarrow\quad
\Exec^{\le C_{\max}}_K \subseteq \mathcal{C}_{\Theta,\gamma}\ \cap\ \mathcal{C}_{\mathrm{stab}}(\Probe_U,\beta),
\]
where $\mathcal{C}_{\Theta,\gamma}$ is the floor contract family (instance-declared) and $\mathcal{C}_{\mathrm{stab}}$ is from Chapter~12 (when present).
\end{definition}

\begin{definition}[Feasibility region]
\label{def:feas-region}
Define the feasibility region at bound $C_{\max}$ as:
\[
\Feas(K,\gamma;C_{\max}) \ :=\ \{\ t\in\mathcal{T}\ :\ \Ach_{K,\gamma}(t,C_{\max})=1\ \}.
\]
\end{definition}

\begin{axiom}[Capacity monotonicity of achievement]
\label{ax:ach-mono-ch13}
If $C_{\max}\preceq C_{\max}'$, then for all $t\in\mathcal{T}$,
\[
\Ach_{K,\gamma}(t,C_{\max})=1\ \Rightarrow\ \Ach_{K,\gamma}(t,C_{\max}')=1.
\]
\end{axiom}

\begin{corollary}[Feasibility regions are upward-closed in capacity]
\label{cor:feas-upward}
If $C_{\max}\preceq C_{\max}'$, then
\[
\Feas(K,\gamma;C_{\max}) \subseteq \Feas(K,\gamma;C_{\max}').
\]
\end{corollary}

\begin{proof}
Immediate from Axiom~\ref{ax:ach-mono-ch13}.
\end{proof}

\subsubsection{Propagation under composition (normative)}

\begin{definition}[Capacity merge operator]
\label{def:cap-merge}
A \emph{capacity merge operator} is a binary operation
\[
\MergeC:\ C\times C \to C
\]
intended to model how costs accumulate under composition.
\end{definition}

\begin{axiom}[Capacity merge associativity (recommended)]
\label{ax:cap-merge-assoc}
For all $c_1,c_2,c_3\in C$,
\[
(c_3\MergeC c_2)\MergeC c_1 = c_3\MergeC (c_2\MergeC c_1).
\]
\end{axiom}

\begin{axiom}[Capacity merge monotonicity (recommended)]
\label{ax:cap-merge-mono}
If $c_1\preceq c_1'$ and $c_2\preceq c_2'$, then
\[
c_1\MergeC c_2 \preceq c_1'\MergeC c_2'.
\]
\end{axiom}

\begin{axiom}[Sequential cost propagation (requires the composition kit)]
\label{ax:cost-prop-seq}
If $\tau_{21}$ is a sequentially composed trace built by the declared constructor kit from $\tau_1,\tau_2$
(i.e.\ $\tau_{21}=\Compose_{\seq}(\tau_1,\tau_2)$), then
\[
\Cost(\tau_{21})=\Cost(\tau_2)\MergeC \Cost(\tau_1).
\]
\end{axiom}

\begin{axiom}[Parallel cost propagation (optional)]
\label{ax:cost-prop-par}
If $\tau^\star$ is a parallelly composed trace built by the declared constructor kit from $\tau_1,\tau_2$
(i.e.\ $\tau^\star=\Compose_{\par}(\tau_1,\tau_2)$), then
\[
\Cost(\tau^\star)=\Cost(\tau_1)\MergeC \Cost(\tau_2).
\]
\end{axiom}

\begin{theorem}[Feasibility propagation for sequential composition (template)]
\label{thm:feas-prop-seq}
Assume:
\begin{enumerate}[leftmargin=*, itemsep=0.2em]
  \item the kernel composition $K_2\seqc K_1$ is defined by the constructor kit (Chapters~4--7),
  \item sequential cost propagation (Axiom~\ref{ax:cost-prop-seq}) and monotonicity (Axiom~\ref{ax:cap-merge-mono}) hold,
  \item there exists an instance-declared \emph{target splitting rule} $S:\mathcal{T}\to \mathcal{T}\times \mathcal{T}$, written
        $S(t)=(t_1,t_2)$, such that achieving $t_1$ for stage $K_1$ and $t_2$ for stage $K_2$ implies achieving $t$ for the composite.
\end{enumerate}
If $t_1\in\Feas(K_1,\gamma;C_1)$ and $t_2\in\Feas(K_2,\gamma';C_2)$ under compatible envelopes
(as required by the sequential compatibility predicate), then
\[
t\in \Feas(K_2\seqc K_1,\gamma;\ C_2\MergeC C_1).
\]
\end{theorem}

\begin{proof}[Proof sketch]
Combine the existence of composed traces (by the $\Sigma 2$-style constructor kit), the cost bound from Axiom~\ref{ax:cost-prop-seq},
and the target-splitting implication that lifts stage achievement to composite achievement.
\end{proof}

\begin{remark}
The only unavoidable instance-specific design choice is $S$ (how end-to-end targets decompose into per-stage targets).
Everything else is structural.
\end{remark}

\subsubsection{Optional: Pareto frontiers (normative option)}

\begin{definition}[Target preorder]
\label{def:target-preorder}
Assume $\mathcal{T}$ carries a preorder $\preceq_{\mathcal{T}}$ expressing ``no worse than.''
For $\mathcal{T}=\ThetaSpace\times\BetaSpace$, a common choice is:
\[
(\Theta,\beta)\preceq_{\mathcal{T}}(\Theta',\beta')
\quad\Longleftrightarrow\quad
\Theta \preceq_\Theta \Theta' \ \wedge\ \beta \preceq_\beta \beta',
\]
with $\preceq_\Theta,\preceq_\beta$ declared by the instance (and their directions made explicit).
\end{definition}

\begin{definition}[Pareto set at capacity]
\label{def:pareto}
Define the Pareto set at bound $C_{\max}$:
\[
\Pareto(K,\gamma;C_{\max})
:= \{\ t\in \Feas(K,\gamma;C_{\max})\ :\ \nexists t'\in \Feas(K,\gamma;C_{\max}) \text{ with } t\prec_{\mathcal{T}} t' \ \}.
\]
\end{definition}

\begin{theorem}[Existence of Pareto optima (sufficient conditions)]
\label{thm:pareto-exists-ch13}
Assume:
\begin{enumerate}[leftmargin=*, itemsep=0.2em]
  \item $\Feas(K,\gamma;C_{\max})$ is nonempty,
  \item $\mathcal{T}$ is finite, or $\mathcal{T}$ is compact and $\Feas(K,\gamma;C_{\max})$ is closed (under a declared topology),
  \item the preorder $\preceq_{\mathcal{T}}$ is compatible with that structure (e.g.\ closed upper sets).
\end{enumerate}
Then $\Pareto(K,\gamma;C_{\max})$ is nonempty.
\end{theorem}

\begin{proof}[Proof sketch]
Standard maximal-element existence: in the finite case, choose a maximal element directly; in the compact/closed case, apply a maximality argument (e.g.\ via upper contour sets).
\end{proof}

\subsubsection{Output checklist (normative)}

To complete the capacity enrichment, a Systemics theory must specify:
\begin{enumerate}[leftmargin=*, itemsep=0.2em]
  \item a capacity poset $(C,\preceq,0_C)$ and a cost functional $\Cost:T^\star\to C$,
  \item the bounded execution trace set $\Exec^{\le C_{\max}}_K$,
  \item a target space $\mathcal{T}$ and achievement predicate $\Ach_{K,\gamma}$, hence feasibility regions $\Feas(K,\gamma;C_{\max})$,
  \item monotonicity of achievement in capacity (Axiom~\ref{ax:ach-mono-ch13}) or an explicit alternative law,
  \item a capacity merge operator $\MergeC$ and (recommended) associativity/monotonicity laws,
  \item propagation axioms (Axiom~\ref{ax:cost-prop-seq}; parallel optional) aligned with the composition kit,
  \item (optional) Pareto notions and sufficient conditions for existence.
\end{enumerate}
