\section{Chapter 4. The Process Algebra Object (normative)}
\label{sec:partII-ch4}

\paragraph{Goal.} Define what a \emph{Systemics process algebra instance} is.

\subsubsection{Optional typed layer (normative)}

\begin{definition}[Type carrier (optional)]
\label{def:Ty}
Optionally, let $\mathsf{Ty}$ be a set of \emph{types}.
\end{definition}

\begin{definition}[Typed kernel (optional)]
\label{def:typed-kernel}
If $\mathsf{Ty}$ is present, a kernel $K$ may be assigned a type (interface)
\[
K:A\to B \quad\text{with}\quad A,B\in \mathsf{Ty}.
\]
This typing is purely declarative at the core level; its operational meaning is induced by
the declared compatibility predicate(s) and composition constructor(s) introduced below.
\end{definition}

\begin{remark}
Typing is optional. When omitted, all kernels are treated as untyped elements of a class $\mathcal{K}$,
and compatibility is expressed entirely at the trace level.
\end{remark}

\subsubsection{Signature of operators (normative)}

\begin{definition}[Kernel carrier]
\label{def:K-carrier}
Let $\mathcal{K}$ be a class (or set) of kernels in the sense of Definition~\ref{def:kernel-primitive}.
\end{definition}

\begin{definition}[Sequential composition operator (signature)]
\label{def:seq-signature}
A \emph{sequential composition operator} is a binary operation on kernels
\[
\seqc:\mathcal{K}\times \mathcal{K}\to \mathcal{K},
\]
written $K_2\seqc K_1$.
\end{definition}

\begin{definition}[Parallel composition operator (optional signature)]
\label{def:par-signature}
Optionally, a \emph{parallel composition operator} is a binary operation
\[
\parc:\mathcal{K}\times \mathcal{K}\to \mathcal{K},
\]
written $K_1\parc K_2$.
\end{definition}

\begin{definition}[Identity kernel (optional signature)]
\label{def:id-signature}
Optionally, for each type $A\in\mathsf{Ty}$ (typed setting) or globally (untyped setting),
there is a distinguished kernel $\mathrm{Id}$ (or $\mathrm{Id}_A$) intended as an identity for $\seqc$.
\end{definition}

\begin{remark}
At this point these operators are just \emph{symbols with arities}. Their semantics is supplied
by the composition kit below (via trace construction).
\end{remark}

\subsubsection{The composition kit (normative)}

The core data that makes the operator symbols meaningful is a trace-level constructor kit.
We package it explicitly.

\begin{definition}[Composition kit]
\label{def:composition-kit}
A \emph{composition kit} is a tuple
\[
\mathfrak{K}_{\mathrm{kit}}=
(\Compat_{\seq},\Compose_{\seq}\ ;\ \Compat_{\par},\Compose_{\par}\ ;\ \Compat_{\mathrm{Id}},\Compose_{\mathrm{Id}})
\]
consisting of:
\begin{itemize}[leftmargin=*, itemsep=0.2em]
  \item a sequential compatibility predicate $\Compat_{\seq}\subseteq T^\star\times T^\star$ and a partial constructor
        $\Compose_{\seq}:T^\star\times T^\star \rightharpoonup T^\star$;
  \item optionally, a parallel compatibility predicate $\Compat_{\par}$ and constructor $\Compose_{\par}$;
  \item optionally, identity compatibility/constructor data $\Compat_{\mathrm{Id}}$ and $\Compose_{\mathrm{Id}}$,
        sufficient to witness identity behavior at the trace level (e.g.\ by providing a canonical trace form).
\end{itemize}
Each declared constructor is required to satisfy the soundness schema (Axiom~\ref{ax:T0}) with its matching predicate.
\end{definition}

\begin{remark}
The kit is where ``what it means to compose'' actually lives. All later composition theorems
(conformance preservation, equivalence congruence, evidence closure) refer back to the kit.
\end{remark}

\subsubsection{Process algebra instance (key deliverable) (normative)}

\begin{definition}[Systemics process algebra instance]
\label{def:process-algebra-instance}
A \emph{Systemics process algebra instance} is a tuple
\[
\mathfrak{P}=
(\mathcal{K},\ \mathsf{Ty};\ \seqc,\ \parc,\ \mathrm{Id};\ \mathfrak{K}_{\mathrm{kit}})
\]
where:
\begin{itemize}[leftmargin=*, itemsep=0.2em]
  \item $\mathcal{K}$ is a kernel carrier (Definition~\ref{def:K-carrier});
  \item $\mathsf{Ty}$ is an optional type carrier (Definition~\ref{def:Ty});
  \item $\seqc$ is a sequential operator (Definition~\ref{def:seq-signature});
  \item $\parc$ is an optional parallel operator (Definition~\ref{def:par-signature});
  \item $\mathrm{Id}$ is an optional identity kernel family (Definition~\ref{def:id-signature});
  \item $\mathfrak{K}_{\mathrm{kit}}$ is a composition kit (Definition~\ref{def:composition-kit}).
\end{itemize}
Additionally, the instance must provide \emph{operator semantics} by specifying, for each declared operator,
how the execution traces of the composite kernel are generated from component traces using the kit; minimally:

\begin{itemize}[leftmargin=*, itemsep=0.2em]
  \item \textbf{Sequential semantics schema:} every trace of $K_2\seqc K_1$ is obtained by choosing
        $\tau_1\in\Exec_{K_1}$ and $\tau_2\in\Exec_{K_2}$ with $\Compat_{\seq}(\tau_1,\tau_2)$ and setting
        \[
        \tau := \Compose_{\seq}(\tau_1,\tau_2)\in \Exec_{K_2\seqc K_1}.
        \]
  \item \textbf{Parallel semantics schema (if present):} analogously using $(\Compat_{\par},\Compose_{\par})$.
  \item \textbf{Identity semantics schema (if present):} identity traces are witnessed by the declared identity kit.
\end{itemize}
\end{definition}

\begin{remark}
Definition~\ref{def:process-algebra-instance} is intentionally ``constructor-first'': it does not assume
algebraic laws (associativity, identities, commutativity). Those become theorems or axioms in later parts
(e.g.\ the equivalence and rewriting chapters) once observational equivalence is defined.
\end{remark}
