\subsection{Meta: Charter, Method, and Reading Guide}
\label{sec:meta}

\subsubsection{Charter (normative)}

\paragraph{What Systemics is.}
Systemics is a \emph{mathematical field} whose central object is a \emph{process}:
a transformation of structured artifacts performed under a declared envelope (regime),
optionally emitting receipts (evidence) and consuming capacity (resources).
Its goal is to provide an \emph{algebra of processes} in which:

\begin{itemize}[leftmargin=*, itemsep=0.25em]
  \item processes can be composed into pipelines,
  \item guarantees (contracts) compose predictably,
  \item processes can be refactored by equational reasoning (rewrite to normal forms),
  \item robustness (stability under probes) is stated and proven,
  \item resource feasibility is treated as a first-class constraint,
  \item executions are \emph{accountable} via proof-like receipts with independent verification.
\end{itemize}

\paragraph{What Systemics is not.}
Systemics is not tied to software, a machine model, a programming language, a proof assistant,
or a particular domain. It may be instantiated in those settings, but the field is defined
at the level of abstract structures, relations, and theorems.

\paragraph{Primary deliverable of the field.}
A Systemics theory is a \emph{coherent package} of:
\begin{enumerate}[leftmargin=*, itemsep=0.2em]
  \item a process algebra (operators and laws),
  \item a semantics class (models) that interprets the algebra as behavior,
  \item contract and evidence layers that make claims compositional and checkable.
\end{enumerate}

\subsubsection{Method: From Core to Enrichments (normative)}

\paragraph{Core-first discipline.}
This treatise is written bottom-up. Each chapter:
\begin{itemize}[leftmargin=*, itemsep=0.2em]
  \item declares its dependencies (which previous definitions/axioms it uses),
  \item introduces no hidden structure,
  \item provides either a theorem or an axiom explicitly.
\end{itemize}

\paragraph{Two-layer development.}
We distinguish:
\begin{itemize}[leftmargin=*, itemsep=0.2em]
  \item \textbf{Core structure:} the minimal objects and laws needed to speak about processes algebraically.
  \item \textbf{Enrichments:} additional structures (stability, capacity, evidence) that attach to the core and remain compositional.
\end{itemize}

\paragraph{Normative vs.\ informative.}
\begin{itemize}[leftmargin=*, itemsep=0.2em]
  \item \textbf{Normative} text defines the field: it contains required definitions, axioms, and theorem targets.
  \item \textbf{Informative} text provides intuition, examples, and optional strengthenings.
\end{itemize}

\subsubsection{Guiding Principles (normative)}

\begin{axiom}[Receipts are witnesses]
A receipt is treated as a mathematical witness object. It may be checked independently of the process that emitted it.
\end{axiom}

\begin{axiom}[Contracts are trace-sets]
A contract is a set (predicate) of admissible traces, and conformance is subset inclusion.
\end{axiom}

\begin{axiom}[Composition alignment]
Whenever a composition operator is introduced on processes, a matching operator must be introduced on contracts and (when present) on receipts/capacities, using the same constructor kit.
\end{axiom}

\begin{remark}
This ``alignment'' principle is the reason Systemics can be both algebraic (equational)
and accountable (proof-relevant): the same constructors drive behavior, guarantees, and evidence.
\end{remark}

\subsubsection{Reading Guide (informative)}

\paragraph{Three readings of the same material.}
This document can be read in three coherent ways:

\begin{enumerate}[leftmargin=*, itemsep=0.25em]
  \item \textbf{Algebra reading (process calculus).}
  Focus on operators, laws, equivalence, rewriting, and normal forms. This is the ``algebra of processes'' track.

  \item \textbf{Semantics reading (models).}
  Focus on execution relations, replay/verification skeletons, and observable views. This is the ``what it means'' track.

  \item \textbf{Accountability reading (evidence + constraints).}
  Focus on stability, capacity feasibility, and receipts-as-proofs with closure under composition.
  This is the ``guarantees you can check'' track.
\end{enumerate}

\paragraph{Minimal path (first pass).}
A recommended minimal first pass is:
\begin{itemize}[leftmargin=*, itemsep=0.2em]
  \item Core objects and contracts (Parts I--II),
  \item execution semantics and observational equivalence (Part III),
  \item conformance preservation theorem (Part II),
  \item evidence closure (Part V).
\end{itemize}

\paragraph{Second pass (calculus maturity).}
Add rewriting + normal forms (Part IV), then stability and capacity (Part V), then exemplars (Part VI).

\subsubsection{What counts as a Systemics Theory? (normative)}

\begin{definition}[Systemics theory]
A \emph{Systemics theory} $\mathcal{S}$ is a tuple
\[
\mathcal{S}=(\mathfrak{P},\mathfrak{M},\mathfrak{Con},\mathfrak{Ev},\mathfrak{Ext})
\]
where:
\begin{itemize}[leftmargin=*, itemsep=0.2em]
  \item $\mathfrak{P}$ is a process algebra instance (kernels + operators + laws, up to equivalence),
  \item $\mathfrak{M}$ is a class of models (semantics) interpreting processes as executions/traces,
  \item $\mathfrak{Con}$ is a contract algebra aligned with $\mathfrak{P}$ and preserved by composition,
  \item $\mathfrak{Ev}$ is an evidence layer (receipts + verifier) with closure under composition,
  \item $\mathfrak{Ext}$ is an (optional) set of enrichments (stability, capacity, morphisms) that remain aligned.
\end{itemize}
\end{definition}

\begin{remark}
Different Systemics theories share the same \emph{shape} but differ in what they take as primitives:
what $U$ is, what receipts look like, how observables are chosen, what stability probes mean, and how capacity is ordered.
\end{remark}

\subsubsection{Compatibility with Existing Drafts (informative)}

The prior $\Sigma 1$--$\Sigma 6$ drafts correspond to chapters as follows:
\begin{itemize}[leftmargin=*, itemsep=0.2em]
  \item $\Sigma 1$ $\to$ execution/receipts/replay skeleton (Part III),
  \item $\Sigma 2$ $\to$ composition + conformance preservation (Part II),
  \item $\Sigma 3$ $\to$ equivalence + rewriting + normal forms (Parts III--IV),
  \item $\Sigma 4$ $\to$ stability enrichment (Part V),
  \item $\Sigma 5$ $\to$ capacity enrichment (Part V),
  \item $\Sigma 6$ $\to$ evidence layer (Part V).
\end{itemize}

\subsubsection{Field Boundary and Ambition (informative)}

Systemics aims to be ``foundational'' in the same sense that:
\begin{itemize}[leftmargin=*, itemsep=0.2em]
  \item algebra abstracts arithmetic and structure-preserving transformation,
  \item category theory abstracts compositional structure,
  \item logic abstracts derivability and proof,
\end{itemize}
but it is centered on \emph{process} (transformation under constraints) with explicit accountability objects (receipts)
and explicit robustness/resource structure (budgets/capacity).

\paragraph{Pragmatic criterion.}
Systemics becomes a field (not a vocabulary) when:
\begin{itemize}[leftmargin=*, itemsep=0.2em]
  \item multiple nontrivial theorems are reusable across domains,
  \item exemplars instantiate the same axioms/laws with minimal ad hoc additions,
  \item normal forms enable real refactorability,
  \item verification closure makes evidence compositional and checkable.
\end{itemize}
