% book/chapters/ch02-universes-judgments.tex

\subsection{Chapter 2. Primitive Universes and Judgments (normative)}
\label{sec:partI-ch2}

\paragraph{Goal.} Fix the minimal typed universe in which everything is stated.

\subsubsection{Primitive carrier sets (normative)}

\begin{definition}[Artifact universe]
\label{def:U}
Let $U$ be a set. Elements $u\in U$ are called \emph{artifacts}.
\end{definition}

\begin{definition}[Valuation space]
\label{def:V}
Let $V$ be a set. Elements $v\in V$ are called \emph{valuations}.
\end{definition}

\begin{definition}[Decision space]
\label{def:2}
Let $2 := \{0,1\}$ be the set of \emph{decisions}.
\end{definition}

\begin{remark}
No structure is assumed on $U$ or $V$ at this stage (no topology, metric, algebra). Those are enrichments
introduced later if needed.
\end{remark}

\subsubsection{Context and evidence carriers (normative)}

\begin{definition}[Envelope / regime context]
\label{def:Gamma}
Let $\Gamma$ be a set. Elements $\gamma\in \Gamma$ are called \emph{envelopes} (or \emph{regimes}).
An envelope abstracts policy/epoch/mode constraints relevant to interpretation and verification.
\end{definition}

\begin{definition}[Receipt space]
\label{def:R}
Let $R$ be a set. Elements $r\in R$ are called \emph{receipts}.
\end{definition}

\begin{definition}[Capacity space (optional)]
\label{def:C}
Optionally, let $C$ be a set whose elements are called \emph{capacity witnesses} (or \emph{cost witnesses}).
\end{definition}

\begin{remark}
At the core level, $C$ is merely a carrier set. Order/monoid structure (for feasibility and propagation)
is introduced in the capacity chapter.
\end{remark}

\subsubsection{Execution parameterization (normative)}

\begin{definition}[Execution parameter index set]
\label{def:T}
Let $T$ be a set. Elements $t\in T$ are called \emph{execution parameters}.
Intuitively, $t$ may represent an index, a schedule selector, a seed, or any abstract knob needed
to distinguish executions; Systemics does not interpret $T$ internally.
\end{definition}

\subsubsection{Trace schema as a witness type (normative)}

\begin{definition}[Trace type]
\label{def:trace-type}
Define the \emph{trace type} (witness-object type)
\[
T^\star \ :=\ (T\times U\times \Gamma)\times (U\times V\times 2\times R\times C\times \Gamma),
\]
when $C$ is present, and otherwise
\[
T^\star \ :=\ (T\times U\times \Gamma)\times (U\times V\times 2\times R\times \Gamma).
\]
We write a trace $\tau\in T^\star$ as
\[
\tau = (t,u,\gamma;\ u',v',d',r',c',\gamma')
\]
(with the $c'$ component omitted when $C$ is absent).
\end{definition}

\begin{definition}[Trace projections]
\label{def:trace-proj}
For $\tau=(t,u,\gamma;\ u',v',d',r',c',\gamma')\in T^\star$, define the projections:
\[
\mathrm{in}(\tau):=(t,u,\gamma),\qquad
\mathrm{out}(\tau):=(u',v',d',r',c',\gamma'),
\]
and component projections:
\[
t(\tau):=t,\quad u(\tau):=u,\quad \gamma(\tau):=\gamma,\quad
u'(\tau):=u',\quad v(\tau):=v',\quad d(\tau):=d',\quad r(\tau):=r',\quad c(\tau):=c',\quad \gamma'(\tau):=\gamma'.
\]
(If $C$ is absent, omit $c(\tau)$.)
\end{definition}

\begin{remark}
Traces are the \emph{witness objects} of the field: contracts, conformance, equivalence, rewriting,
stability, capacity feasibility, and evidence all ultimately talk about sets of traces or maps out of traces.
\end{remark}

\subsubsection{Observables and claims (normative outputs)}

\begin{definition}[Observable carrier (optional)]
\label{def:obs-carrier}
Optionally, fix a set $\mathcal{O}$ called the \emph{observable carrier} and a map
\[
\mathrm{obs}:T^\star \to \mathcal{O},
\]
intended to forget non-observable parts of a trace (e.g.\ hide internal receipt details or intermediate artifacts).
\end{definition}

\begin{remark}
If $\mathcal{O}$ is not fixed here, it may be introduced later (e.g.\ in the equivalence chapter)
as a derived object such as the canonical view $\cview$.
\end{remark}

\begin{definition}[Claim shape (minimal)]
\label{def:claim-shape}
A \emph{claim} is any element of a designated set $\mathsf{Claim}$ whose members are intended to be
witnessable by receipts and/or traces. Minimally, a claim may be taken as a record with fields:
\[
\phi = (\gamma,\ t,\ u,\ u',\ v,\ d),
\]
optionally extended with receipt and capacity fields:
\[
\phi = (\gamma,\ t,\ u,\ u',\ v,\ d,\ r,\ c,\ \gamma').
\]
The choice of $\mathsf{Claim}$ is left abstract at the core level; later chapters constrain it
when introducing verification and evidence closure.
\end{definition}

\paragraph{Chapter outputs (summary).}
Chapter~\ref{sec:partI-ch2} fixes the carrier sets $U,V,2,\Gamma,R$ (and optional $C$), the parameter set $T$,
the witness type $T^\star$, and the basic projections $\mathrm{in}(\cdot)$ and $\mathrm{out}(\cdot)$.
