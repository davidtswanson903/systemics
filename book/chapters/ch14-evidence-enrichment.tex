% book/chapters/ch14-evidence-enrichment.tex

\subsection{Chapter 14. Evidence Enrichment (normative)}
\label{sec:partV-ch14}

\paragraph{Goal.} Receipts as proof objects with merge-closed verification.

\subsubsection{Claim space and claim extraction (normative)}

\begin{definition}[Claim space]
\label{def:claim-space}
Let $\mathsf{Claim}$ be a set of propositions (claims) that receipts may witness.
A minimal claim schema may include:
\[
\phi=(\gamma,\Theta,\beta,C_{\max},\mathrm{in},\mathrm{out},v,d),
\]
asserting that under envelope $\gamma$, targets $(\Theta,\beta)$, and capacity bound $C_{\max}$,
a given input produces a given output/valuation/decision in the intended semantics.
\end{definition}

\begin{definition}[Claim extraction]
\label{def:claim-extraction}
A \emph{claim extraction map} is a function
\[
\Phi:\ T^\star \to \mathsf{Claim},
\]
assigning to each trace $\tau$ the claim it purports to witness.
\end{definition}

\begin{remark}
$\Phi$ is part of the mathematical interface of a Systemics theory: it declares what proposition is being proven by a trace's receipt.
\end{remark}

\subsubsection{Receipts as proof objects (normative)}

\begin{definition}[Receipt entailment relation]
\label{def:receipt-entails}
An \emph{entailment intent} relation is a relation
\[
\Vdash\ \subseteq\ R\times \mathsf{Claim},
\]
where $r\Vdash \phi$ means ``$r$ is intended as a proof object for claim $\phi$.''
\end{definition}

\begin{definition}[Verifier]
\label{def:verifier}
A \emph{verifier} is a predicate
\[
\Verify:\ R\times \mathsf{Claim}\to 2,
\]
where $\Verify(r,\phi)=1$ means ``$r$ verifies claim $\phi$.'' The verifier is required to be
\emph{independent} of the kernel's internal implementation: it depends only on the declared semantics and the receipt schema.
\end{definition}

\subsubsection{Verifier invariants (normative)}

\begin{axiom}[E0: Canonical invariance of verification]
\label{ax:E0-verify-canon}
For all $r\in R$ and all $\phi\in \mathsf{Claim}$,
\[
\Verify(r,\phi)=\Verify(\canon(r),\phi).
\]
\end{axiom}

\begin{axiom}[E1: Binding consistency]
\label{ax:E1-verify-bind}
If $\Verify(r,\phi)=1$, then $\bind(\canon(r))$ is defined and agrees with the envelope component of $\phi$
on the declared binding fields (e.g.\ epoch/regime identifier).
\end{axiom}

\begin{remark}
Axiom~\ref{ax:E1-verify-bind} prevents ``receipt laundering'' across regimes: a receipt cannot verify a claim for an unrelated envelope.
\end{remark}

\subsubsection{Composed claims and evidence closure (normative)}

\begin{definition}[Claim compatibility (sequential)]
\label{def:claim-compat}
Two claims $\phi_1,\phi_2\in\mathsf{Claim}$ are \emph{sequentially compatible}, written $\Compat_\phi(\phi_1,\phi_2)$,
if the output interface fields of $\phi_1$ match the input interface fields of $\phi_2$ in the declared sense
(e.g.\ output artifact equals next input artifact, and final envelope matches next initial envelope on binding fields).
\end{definition}

\begin{definition}[Claim composition (sequential)]
\label{def:claim-compose}
Given compatible claims $\Compat_\phi(\phi_1,\phi_2)$, define a composed claim
\[
\phi_{21}:=\phi_2\diamond \phi_1,
\]
whose components are determined by the declared composition mode (last-wins, product, aggregated) and the declared
capacity/target propagation rule (e.g.\ $C_{\max,21}=C_{\max,2}\MergeC C_{\max,1}$ when applicable).
\end{definition}

\begin{definition}[Merge-closed verification (sequential)]
\label{def:merge-closed-verify}
Fix a receipt merge operator $\MergeR:R\times R\to R$ (from the composition kit) and define merged receipts by
$r_{21}:=\canon(r_2\MergeR r_1)$.
The verifier is \emph{merge-closed for sequential composition} if whenever
\[
\Verify(r_1,\phi_1)=1,\quad \Verify(r_2,\phi_2)=1,\quad \Compat_\phi(\phi_1,\phi_2),
\]
then the verifier can (from $r_{21}$) validate (i) the presence/references of subproofs for $r_1,r_2$, and (ii) the interface compatibility checks,
and thereby conclude $\Verify(r_{21},\phi_2\diamond \phi_1)=1$.
\end{definition}

\begin{theorem}[Evidence closure under sequential composition]
\label{thm:evidence-closure}
Assume merge-closed verification (Definition~\ref{def:merge-closed-verify}). Then for all receipts and claims:
\[
\Verify(r_1,\phi_1)=1\ \wedge\ \Verify(r_2,\phi_2)=1\ \wedge\ \Compat_\phi(\phi_1,\phi_2)
\ \Rightarrow\ 
\Verify\!\big(\canon(r_2\MergeR r_1),\ \phi_2\diamond \phi_1\big)=1.
\]
\end{theorem}

\begin{proof}[Proof sketch]
By merge-closed verification, $\canon(r_2\MergeR r_1)$ is checkable as a structured aggregate of the two subproofs and the compatibility witness.
Since each subproof verifies its claim and the compatibility checks pass, the composed claim verifies.
Axioms~\ref{ax:E0-verify-canon}--\ref{ax:E1-verify-bind} ensure canonical stability and envelope coherence.
\end{proof}

\begin{remark}
A parallel version is obtained by replacing $\diamond$ with the parallel claim constructor (when $\parc$ is adopted) and by using the parallel trace/receipt constructor kit.
\end{remark}

\subsubsection{Proof-carrying conformance (normative)}

\begin{definition}[Proof-carrying conformance contract]
\label{def:pc-contract}
Define the \emph{proof-carrying conformance contract} $\mathcal{C}_{\mathrm{pc}}\subseteq T^\star$ by:
\[
\tau\in \mathcal{C}_{\mathrm{pc}}
\quad:\Longleftrightarrow\quad
\Verify\!\big(r(\tau),\,\Phi(\tau)\big)=1,
\]
where $r(\tau)$ is the receipt component of $\tau$ and $\Phi$ is the claim extraction map (Definition~\ref{def:claim-extraction}).
\end{definition}

\begin{theorem}[Proof-carrying execution]
\label{thm:pc-exec}
If a kernel $K$ conforms to $\mathcal{C}_{\mathrm{pc}}$ (in the Chapter~6 sense), then every execution trace of $K$
is accompanied by independently verifiable evidence of its declared claim.
\end{theorem}

\begin{proof}
Immediate from Definition~\ref{def:pc-contract} and the definition of conformance as subset inclusion.
\end{proof}

\subsubsection{Output checklist (normative)}

To complete the evidence enrichment, a Systemics theory must specify:
\begin{enumerate}[leftmargin=*, itemsep=0.2em]
  \item a claim space $\mathsf{Claim}$ and a claim extraction map $\Phi:T^\star\to\mathsf{Claim}$,
  \item an entailment intent relation $\Vdash \subseteq R\times \mathsf{Claim}$,
  \item a verifier predicate $\Verify:R\times\mathsf{Claim}\to 2$ satisfying Axioms~\ref{ax:E0-verify-canon}--\ref{ax:E1-verify-bind},
  \item claim compatibility and composition rules (Definition~\ref{def:claim-compat}--\ref{def:claim-compose}),
  \item merge-closed verification (Definition~\ref{def:merge-closed-verify}) or explicit alternative composition semantics,
  \item the proof-carrying conformance contract $\mathcal{C}_{\mathrm{pc}}$.
\end{enumerate}
