\section{Chapter 8. Replay and Verification Skeleton (normative)}
\label{sec:partIII-ch8}

\paragraph{Goal.} Define replay/verification \emph{mathematically} without committing to implementation.

\subsubsection{Receipt canonicalization and binding (normative)}

\begin{definition}[Receipt canonicalization]
\label{def:canon-ch8}
A \emph{canonicalization map} is a function
\[
\canon:R\to R.
\]
\end{definition}

\begin{axiom}[R0: Canonicalization idempotence]
\label{ax:R0-idem}
For all $r\in R$,
\[
\canon(\canon(r))=\canon(r).
\]
\end{axiom}

\begin{definition}[Envelope binding (partial)]
\label{def:bind}
An \emph{envelope binding map} is a partial function
\[
\bind:R \rightharpoonup \Gamma,
\]
intended to recover (at least) the effective regime/epoch from a receipt.
\end{definition}

\begin{axiom}[B0: Binding coherence with execution]
\label{ax:B0-bind-coh}
There exists a designated set of \emph{binding fields} of envelopes (instance-defined) such that:
for every execution witness
\[
(t,u,\gamma)\xRightarrow{K}(u',v',d',r',c',\gamma'),
\]
the value $\bind(\canon(r'))$ is defined and agrees with $\gamma$ on the declared binding fields.
\end{axiom}

\begin{remark}
Axiom~\ref{ax:B0-bind-coh} does \emph{not} require $\bind(\canon(r'))=\gamma$ in full; it requires agreement on
those components the theory declares to be binding-relevant (e.g.\ epoch, policy-id, regime tag, version).
\end{remark}

\subsubsection{Replay relation (normative)}

\begin{definition}[Replay relation (partial)]
\label{def:replay-rel}
A \emph{replay relation} is a relation
\[
\Replay\ \subseteq\ (R\times U)\times (V\times 2).
\]
We write
\[
(r,u)\Replay (v,d)
\]
to mean: ``replaying receipt $r$ on artifact $u$ yields valuation $v$ and decision $d$.''

Replay is allowed to be partial: for some $(r,u)$ there may be no related $(v,d)$.
\end{definition}

\begin{remark}
Replay is intentionally detached from kernel internals. It is a mathematical relation that
connects (receipt, artifact) pairs to (valuation, decision) outputs.
\end{remark}

\subsubsection{Verification predicate (minimum interface) (normative)}

\begin{definition}[Verification predicate (minimum interface)]
\label{def:verify-min}
A \emph{verification predicate} is a map returning a decision bit:
\[
\Verify:\ R\times U\times \Gamma \times \ThetaSpace \times \BetaSpace \times C \ \to\ 2,
\]
where $\ThetaSpace$ and $\BetaSpace$ are the carriers for floor settings and invariance budgets (when present).
In contexts where $(\Theta,\beta,C)$ are not explicit, the corresponding arguments may be omitted, yielding the minimal form:
\[
\Verify:\ R\times U\times \Gamma \to 2.
\]
\end{definition}

\begin{remark}
The purpose of $\Verify$ here is not to prescribe \emph{how} verification is done, but to insist that
verification is a well-typed mathematical judgment, potentially parameterized by envelope and targets.
\end{remark}

\subsubsection{Minimal theorem targets (normative)}

\begin{definition}[Replay determinism on a domain]
\label{def:replay-det-domain}
Let $D\subseteq R\times U$. Replay is \emph{deterministic on $D$} if for all $(r,u)\in D$:
\[
(r,u)\Replay(v_1,d_1)\ \wedge\ (r,u)\Replay(v_2,d_2)
\quad\Rightarrow\quad
(v_1,d_1)=(v_2,d_2).
\]
\end{definition}

\begin{theorem}[Replay determinism under canonical receipts (target)]
\label{thm:replay-det-canon}
Assume there exists a domain $D\subseteq R\times U$ such that:
\begin{enumerate}[leftmargin=*, itemsep=0.2em]
  \item (\emph{canonical definedness}) for all $(r,u)\in D$, there exists $(v,d)$ with $(\canon(r),u)\Replay(v,d)$;
  \item (\emph{canonical determinism}) replay is deterministic on $\{(\canon(r),u):(r,u)\in D\}$ in the sense of
        Definition~\ref{def:replay-det-domain}.
\end{enumerate}
Then canonical receipts support deterministic replay on $D$: for all $(r,u)\in D$ there exists a unique $(v,d)$ with
\[
(\canon(r),u)\Replay(v,d).
\]
\end{theorem}

\begin{proof}
By (1) replay is defined on $(\canon(r),u)$ for each $(r,u)\in D$; by (2) any two replay outcomes must be equal,
so the outcome is unique.
\end{proof}

\begin{remark}
Theorem~\ref{thm:replay-det-canon} is phrased as a \emph{target} because the sufficient hypotheses are instance-dependent.
Many instances enforce it by designing receipts whose canonical form fixes all replay-relevant choices.
\end{remark}k/chapters/ch08-replay-verification.tex

\subsection{Chapter 8. Replay and Verification Skeleton}

% Placeholder. (Source content currently lives under tex/chapters/.)
