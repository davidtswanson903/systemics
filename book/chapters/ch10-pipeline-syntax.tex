\section{Chapter 10. Syntax of Pipelines (normative)}
\label{sec:partIV-ch10}

\paragraph{Goal.} Introduce expressions so we can do algebraic rewriting.

\subsubsection{Expression language (syntax) (normative)}

\begin{definition}[Atomic kernel symbols]
\label{def:atomic-kernels}
Let $\mathcal{K}_0$ be a set of \emph{atomic kernel symbols}. Elements of $\mathcal{K}_0$ are syntactic names (not semantics).
\end{definition}

\begin{definition}[Pipeline expression grammar]
\label{def:expr-grammar}
Define the set of \emph{pipeline expressions} $\mathcal{E}$ inductively by:
\begin{itemize}[leftmargin=*, itemsep=0.15em]
  \item (\textbf{atoms}) if $K\in\mathcal{K}_0$ then $K\in\mathcal{E}$;
  \item (\textbf{sequential}) if $E_1,E_2\in\mathcal{E}$ then $(E_2\seqc E_1)\in\mathcal{E}$;
  \item (\textbf{parallel}) if $E_1,E_2\in\mathcal{E}$ then $(E_1\parc E_2)\in\mathcal{E}$;
  \item (\textbf{branch}) if $E_L,E_R\in\mathcal{E}$ and $s$ is a selector (instance-declared),
        then $\mathrm{branch}(E_L,E_R,s)\in\mathcal{E}$.
\end{itemize}
\end{definition}

\begin{remark}
The symbols $\seqc,\parc,\mathrm{branch}$ are \emph{syntactic constructors} in this chapter.
Their meaning is assigned by interpretation (next subsection) using the process algebra instance (Chapter~4) and the semantic model (Chapter~7).
\end{remark}

\subsubsection{Interpretation into kernels/semantics (normative)}

\begin{definition}[Interpretation function]
\label{def:interp}
An \emph{interpretation} is a function
\[
\llbracket\cdot\rrbracket:\ \mathcal{E}\to \mathcal{K},
\]
mapping expressions to (semantic) kernels in the ambient Systemics process algebra instance.
It is defined structurally by:
\begin{itemize}[leftmargin=*, itemsep=0.2em]
  \item $\llbracket K\rrbracket := K$ for $K\in\mathcal{K}_0\subseteq \mathcal{K}$ (atoms denote kernels);
  \item $\llbracket E_2\seqc E_1\rrbracket := \llbracket E_2\rrbracket \seqc \llbracket E_1\rrbracket$;
  \item $\llbracket E_1\parc E_2\rrbracket := \llbracket E_1\rrbracket \parc \llbracket E_2\rrbracket$;
  \item $\llbracket \mathrm{branch}(E_L,E_R,s)\rrbracket := \mathrm{branch}(\llbracket E_L\rrbracket,\llbracket E_R\rrbracket,s)$,
        where $\mathrm{branch}(\cdot)$ is the (optional) branching operator from the composition kit.
\end{itemize}
\end{definition}

\begin{remark}
Definition~\ref{def:interp} makes the expression language a \emph{free} syntax generated by $\mathcal{K}_0$
and the constructors, with semantics supplied by the chosen algebra instance.
\end{remark}

\subsubsection{Semantic meaning of an expression (normative)}

\begin{definition}[Expression semantics via relational execution]
\label{def:expr-semantics}
Given the relational execution semantics $\Exec_K$ for each kernel (Chapter~7), define the semantics of an expression $E\in\mathcal{E}$ by:
\[
\Exec_E \ :=\ \Exec_{\llbracket E\rrbracket}.
\]
Equivalently, the traces of $E$ are the traces of $\llbracket E\rrbracket$.
\end{definition}

\begin{remark}
This definition is intentionally minimal: rewriting acts on expressions, while correctness is stated in terms of
kernel equivalence $\simeq$ (Chapter~9) applied to $\llbracket\cdot\rrbracket$.
\end{remark}
