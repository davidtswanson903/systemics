% book/chapters/ch15-exemplar-method.tex

\subsection{Chapter 15. Exemplar Method (normative template)}
\label{sec:exemplar-method}

\paragraph{Goal.} Define what counts as a canonical exemplar for the field.

\subsubsection{Exemplar object (normative)}

\begin{definition}[Canonical exemplar]
\label{def:canonical-exemplar}
A \emph{canonical exemplar} is a tuple
\[
\mathbb{E}=(\mathbb{U},\ \mathbb{K},\ \mathbb{A},\ \mathbb{O},\ \mathbb{X}),
\]
where:
\begin{itemize}[leftmargin=*, itemsep=0.15em]
  \item $\mathbb{U}$ (\textbf{Universes}) instantiates the primitive carriers from Chapter~2:
  \[
  (U,V,2,\Gamma,R,T)\ \text{ and trace type }T^\star;
  \]
  optionally a capacity space $(C,\preceq)$.
  \item $\mathbb{K}$ (\textbf{Kernel family}) specifies a nonempty set of kernels (atomic generators) together with their execution semantics $\Exec_K$ (Chapter~7),
  and the associated trace fields (including receipts).
  \item $\mathbb{A}$ (\textbf{Composition kit / process algebra instance}) instantiates Chapter~4:
  \[
  (\seqc,\parc,\mathrm{Id},\Compat,\Compose_{\seq},\Compose_{\par},\MergeR,\MergeC,\ldots),
  \]
  with any optional components explicitly marked as optional.
  \item $\mathbb{O}$ (\textbf{Observables \& induced equivalence}) instantiates Chapter~9:
  an observable view $\cview:T^\star\to\mathcal{O}$, trace equivalence $\approx$, and induced kernel equivalence $\simeq$ (with congruence conditions).
  \item $\mathbb{X}$ (\textbf{Optional enrichments}) instantiates at least one enrichment chapter (Part~V), e.g.\
  stability (Chapter~12), capacity (Chapter~13), or evidence (Chapter~14), with all required structures.
\end{itemize}
\end{definition}

\begin{remark}
The tuple notation is only packaging: any exemplar is acceptable if it presents the same data unambiguously and checks the same obligations.
\end{remark}

\subsubsection{Exemplar obligations (normative)}

\begin{definition}[Exemplar obligations]
\label{def:exemplar-obligations}
A candidate exemplar $\mathbb{E}$ is \emph{admissible} as canonical iff it satisfies all obligations below.

\paragraph{(E0) Instantiation completeness.}
$\mathbb{E}$ explicitly instantiates:
\begin{enumerate}[leftmargin=*, itemsep=0.15em]
  \item universes and trace schema (Chapter~2),
  \item the process algebra instance / composition kit (Chapter~4),
  \item observational interface ($\cview$, $\approx$, $\simeq$) (Chapter~9),
  \item at least one enrichment package (Chapters~12--14).
\end{enumerate}

\paragraph{(E1) Conformance preservation demonstration.}
$\mathbb{E}$ must include at least one explicit theorem instance of Chapter~6:
there exist contracts $\mathcal{C}_1,\mathcal{C}_2$ and kernels $K_1,K_2$ such that:
\[
K_1\models\mathcal{C}_1,\quad K_2\models\mathcal{C}_2
\quad\Rightarrow\quad
K_2\seqc K_1 \models (\mathcal{C}_2\cseq \mathcal{C}_1),
\]
with all compatibility and constructor conditions checked in the instance.

\paragraph{(E2) Equivalence/normalization demonstration.}
$\mathbb{E}$ must exhibit at least one nontrivial refactoring proof using Chapter~11:
there exist expressions $E,E'$ with
\[
E \RewriteStar E'
\quad\text{and}\quad
\llbracket E\rrbracket \simeq \llbracket E'\rrbracket,
\]
and either:
\begin{itemize}[leftmargin=*, itemsep=0.15em]
  \item an explicit normalization function $\Norm$ with soundness, or
  \item an explicit normal-form predicate $\NF$ plus a proof that the chosen refactoring reaches $\NF$ and preserves $\simeq$.
\end{itemize}

\paragraph{(E3) Enrichment demonstration.}
$\mathbb{E}$ must demonstrate at least one enrichment as an explicit theorem instance:
\begin{itemize}[leftmargin=*, itemsep=0.15em]
  \item \textbf{Stability:} a probe family and a wobble structure yielding a stated invariance result, or
  \item \textbf{Capacity:} a feasibility region statement with propagation under $\MergeC$, or
  \item \textbf{Evidence:} an evidence closure statement for merged receipts verifying composed claims.
\end{itemize}

\paragraph{(E4) Verifier-checkability (evidence interface).}
If the exemplar uses receipts (it must, by Chapters~2 and~7), then it must specify a verifier interface
at the level of Chapter~8/14 such that the exemplar's receipt claims are \emph{checkable}:
either
\begin{itemize}[leftmargin=*, itemsep=0.15em]
  \item a concrete verifier predicate $\Verify(r,\phi)$ is given, or
  \item an abstract verifier is given with explicitly stated axioms sufficient to justify the exemplar's verification steps,
        including canonical invariance and binding consistency.
\end{itemize}
In particular, if evidence enrichment (Chapter~14) is the chosen enrichment, the exemplar must satisfy merge-closed verification.

\end{definition}

\subsubsection{Canonical exemplar output format (normative)}

\begin{definition}[Exemplar dossier]
\label{def:exemplar-dossier}
A canonical exemplar must be presented as an \emph{exemplar dossier} containing:
\begin{enumerate}[leftmargin=*, itemsep=0.2em]
  \item \textbf{Instantiation page:} explicit definitions for $\mathbb{U}$, $\mathbb{K}$, $\mathbb{A}$, $\mathbb{O}$, and chosen $\mathbb{X}$.
  \item \textbf{Obligation proofs:} proofs (or proof sketches labeled as such) satisfying (E1)--(E4) of Definition~\ref{def:exemplar-obligations}.
  \item \textbf{Receipts/verifier note:} a statement of the receipt schema and verifier interface used, including binding fields and canonicalization.
  \item \textbf{Refactoring exhibit:} at least one normalization/refactoring example with the $\simeq$ witness.
\end{enumerate}
\end{definition}

\begin{remark}
The exemplar dossier is the field-level unit of pedagogy and validation: it is what can be cited, replayed abstractly, and compared across domains.
\end{remark}
