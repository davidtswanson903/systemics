% tex/chapters/ch09-observables-eq.tex

\subsection{Chapter 9. Observables and Induced Equivalence (normative)}
\label{sec:partIII-ch9}

\paragraph{Goal.} Define \emph{what counts as the same process} for refactoring.

\subsubsection{Observable carrier and view map (normative)}

\begin{definition}[Observable carrier]
\label{def:obs-carrier}
An \emph{observable carrier} is a set $\mathcal{O}$ whose elements represent the externally relevant view of traces.
\end{definition}

\begin{definition}[Observable view map]
\label{def:cview}
An \emph{observable view map} is a function
\[
\cview:\ T^\star \to \mathcal{O}.
\]
\end{definition}

\begin{remark}
$\cview$ is the formal ``information boundary.'' It specifies what is visible to equivalence.
For example, $\cview$ may:
\begin{itemize}
  \item forget raw receipts and retain only canonical digests,
  \item forget intermediate states and retain only input/output summaries,
  \item include envelope binding fields but omit non-binding fields.
\end{itemize}
All such choices are mathematical and instance-declared.
\end{remark}

\subsubsection{Trace equivalence induced by observables (normative)}

\begin{definition}[Trace observational equivalence]
\label{def:trace-eq}
For traces $\tau,\tilde{\tau}\in T^\star$, define
\[
\tau \approx \tilde{\tau}
\quad:\Longleftrightarrow\quad
\cview(\tau)=\cview(\tilde{\tau}).
\]
\end{definition}

\begin{lemma}[$\approx$ is an equivalence relation]
\label{lem:trace-eq-rel}
The relation $\approx$ is reflexive, symmetric, and transitive.
\end{lemma}

\begin{proof}
Immediate from the fact that $=$ on $\mathcal{O}$ is an equivalence relation and $\approx$ is defined as equality of images under $\cview$.
\end{proof}

\subsubsection{Kernel equivalence induced by trace matching (normative)}

Recall that in the relational model class (Chapter~7), for each kernel $K$ the semantics $\Exec_K$ induces a set of traces
$\widetilde{\Exec}_K\subseteq T^\star$ (Definition~\ref{def:trace-induced}).
For notational convenience, write
\[
\Exec^\star_K := \widetilde{\Exec}_K.
\]

\begin{definition}[Kernel observational equivalence induced by $\approx$]
\label{def:kernel-eq}
Two kernels $K$ and $K'$ are \emph{observationally equivalent} (written $K\simeq K'$) if for all execution inputs $(t,u,\gamma)$:
\begin{enumerate}
  \item for every trace $\tau\in \Exec^\star_K$ with $\mathrm{in}(\tau)=(t,u,\gamma)$, there exists a trace
        $\tilde{\tau}\in \Exec^\star_{K'}$ with $\mathrm{in}(\tilde{\tau})=(t,u,\gamma)$ such that $\tau\approx \tilde{\tau}$;
  \item conversely, for every trace $\tilde{\tau}\in \Exec^\star_{K'}$ with input $(t,u,\gamma)$, there exists
        $\tau\in \Exec^\star_K$ with the same input such that $\tau\approx \tilde{\tau}$.
\end{enumerate}
\end{definition}

\begin{remark}
Definition~\ref{def:kernel-eq} is a bisimulation-style equivalence on \emph{sets of observable traces}.
If kernels are deterministic, then for each input there is (at most) one trace, and $K\simeq K'$ reduces to equality of observable views for that trace.
\end{remark}

\subsubsection{$\simeq$ is an equivalence relation (normative)}

\begin{theorem}[$\simeq$ is an equivalence relation]
\label{thm:kernel-eq-rel}
The relation $\simeq$ is reflexive, symmetric, and transitive.
\end{theorem}

\begin{proof}
Reflexive: for any $K$, match each trace to itself; since $\approx$ is reflexive (Lemma~\ref{lem:trace-eq-rel}), the witness exists.
Symmetric: the definition is symmetric by swapping $K$ and $K'$.
Transitive: assume $K\simeq K'$ and $K'\simeq K''$. Fix an input $(t,u,\gamma)$ and a trace $\tau$ of $K$ at that input.
By $K\simeq K'$, there exists $\tau'$ of $K'$ at the same input with $\tau\approx \tau'$.
By $K'\simeq K''$, there exists $\tau''$ of $K''$ at the same input with $\tau'\approx \tau''$.
By transitivity of $\approx$ (Lemma~\ref{lem:trace-eq-rel}), $\tau\approx \tau''$. The reverse direction is analogous.
\end{proof}

\subsubsection{Compositional observables and congruence (normative)}

To ensure refactoring is stable under composition, we require that observables respect the composition kit.

\begin{definition}[Compositional observable view (sequential)]
\label{def:obs-compositional}
The observable view $\cview$ is \emph{sequentially compositional} if there exists a (partial) operator
\[
\odot_{\seq}:\ \mathcal{O}\times \mathcal{O} \rightharpoonup \mathcal{O}
\]
such that whenever $\Compat_{\seq}(\tau_1,\tau_2)$ holds and $\tau_{21}=\Compose_{\seq}(\tau_1,\tau_2)$ is defined,
we have
\[
\cview(\tau_{21}) \ =\ \cview(\tau_2)\ \odot_{\seq}\ \cview(\tau_1).
\]
\end{definition}

\begin{remark}
This is the formal statement of ``observables compose by the same recipe.'' In many instances,
$\odot_{\seq}$ is induced by canonicalization and merge on the observed components (e.g.\ receipt digests, capacity witnesses, envelope bindings).
\end{remark}

\begin{theorem}[Congruence of $\simeq$ for $\seqc$ under compositional observables]
\label{thm:congruence-seq}
Assume:
\begin{enumerate}
  \item the model axiom M0 holds (Axiom~\ref{ax:M0-seq}),
  \item the observable view is sequentially compositional (Definition~\ref{def:obs-compositional}).
\end{enumerate}
Then $\simeq$ is a congruence for sequential composition:
\[
K_1\simeq K_1'\ \wedge\ K_2\simeq K_2'
\quad\Rightarrow\quad
K_2\seqc K_1 \ \simeq\ K_2'\seqc K_1'.
\]
\end{theorem}

\begin{proof}
Fix an input $(t,u,\gamma)$ and let $\tau$ be any trace of $K_2\seqc K_1$ at that input.
By M0 (Axiom~\ref{ax:M0-seq}), $\tau=\Compose_{\seq}(\tau_1,\tau_2)$ for some compatible stage traces
$\tau_1$ of $K_1$ (at $(t,u,\gamma)$) and $\tau_2$ of $K_2$ (at $(t,u_1,\gamma_1)$ as dictated by $\tau_1$),
with $\Compat_{\seq}(\tau_1,\tau_2)$.

Because $K_1\simeq K_1'$, there exists a trace $\tilde{\tau}_1$ of $K_1'$ matching $\tau_1$ up to $\approx$.
Because $K_2\simeq K_2'$, there exists a trace $\tilde{\tau}_2$ of $K_2'$ matching $\tau_2$ up to $\approx$.
(Here the inputs match as required because $\approx$ equates observable inputs, and the kit-defined compatibility uses declared interface fields.)

Form the composed trace $\tilde{\tau} := \Compose_{\seq}(\tilde{\tau}_1,\tilde{\tau}_2)$ for $K_2'\seqc K_1'$.
By compositionality of $\cview$ (Definition~\ref{def:obs-compositional}),
\[
\cview(\tau) = \cview(\tau_2)\odot_{\seq}\cview(\tau_1)
\quad\text{and}\quad
\cview(\tilde{\tau}) = \cview(\tilde{\tau}_2)\odot_{\seq}\cview(\tilde{\tau}_1).
\]
Since $\tau_1\approx \tilde{\tau}_1$ and $\tau_2\approx \tilde{\tau}_2$, we have
$\cview(\tau_1)=\cview(\tilde{\tau}_1)$ and $\cview(\tau_2)=\cview(\tilde{\tau}_2)$, hence $\cview(\tau)=\cview(\tilde{\tau})$,
i.e.\ $\tau\approx \tilde{\tau}$. The reverse direction is symmetric. Therefore $K_2\seqc K_1 \simeq K_2'\seqc K_1'$.
\end{proof}

\begin{remark}
Theorem~\ref{thm:congruence-seq} is the precise bridge from semantics to the algebraic law L1 (Chapter~5):
it shows congruence is not merely postulated but can be derived from the relational model axiom plus compositional observables.
\end{remark}
