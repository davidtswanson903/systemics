% tex/chapters/ch07-exec-semantics.tex

\subsection{Chapter 7. Execution Semantics (normative)}
\label{sec:partIII-ch7}

\paragraph{Goal.} Provide the canonical \emph{model class} for the algebra: relational execution semantics.

\subsubsection{Relational execution semantics (normative)}

\begin{definition}[Relational execution semantics]
\label{def:exec-relational}
Fix:
\begin{itemize}
  \item an execution parameter set $T$,
  \item an artifact universe $U$,
  \item a context/envelope set $\Gamma$,
  \item a valuation space $V$,
  \item a decision space $2=\{0,1\}$,
  \item a receipt space $R$,
  \item (optionally) a capacity space $C$.
\end{itemize}
A kernel $K$ has \emph{relational execution semantics} given by a relation
\[
\Exec_K\ \subseteq\ (T\times U\times \Gamma)\times (U\times V\times 2\times R\times C\times \Gamma).
\]
We write
\[
(t,u,\gamma)\xRightarrow{K}(u',v',d',r',c',\gamma')
\]
iff $\big((t,u,\gamma),(u',v',d',r',c',\gamma')\big)\in \Exec_K$.
\end{definition}

\begin{remark}
If the theory omits capacity $C$, simply drop the $c'$ coordinate and interpret
\[
\Exec_K\subseteq (T\times U\times\Gamma)\times (U\times V\times 2\times R\times \Gamma).
\]
\end{remark}

\subsubsection{Deterministic semantics as a special case (normative)}

\begin{definition}[Deterministic execution function]
\label{def:exec-deterministic}
A kernel $K$ is \emph{deterministic} (in this semantics) if $\Exec_K$ is single-valued in the output
for each input. In that case, there exists a partial function
\[
\exec_K:\ T\times U\times \Gamma \rightharpoonup U\times V\times 2\times R\times C\times \Gamma
\]
such that
\[
\exec_K(t,u,\gamma)=(u',v',d',r',c',\gamma')
\quad\Longleftrightarrow\quad
(t,u,\gamma)\xRightarrow{K}(u',v',d',r',c',\gamma').
\]
\end{definition}

\begin{remark}
Relational semantics is the canonical model class because it accommodates nondeterminism
(witness choice, admissible decompositions, underspecified joins) while remaining purely mathematical.
Determinism is recovered as a property, not assumed.
\end{remark}

\subsubsection{Trace extraction (normative)}

\begin{definition}[Trace induced by relational semantics]
\label{def:trace-induced}
Given a relation $\Exec_K$ as in Definition~\ref{def:exec-relational}, define the induced trace set
(as an instance of the trace type $T^\star$):
\[
\widetilde{\Exec}_K\ :=\
\big\{\, (t,u,\gamma;\ u',v',d',r',c',\gamma')\ :\ ((t,u,\gamma),(u',v',d',r',c',\gamma'))\in \Exec_K\,\big\}.
\]
\end{definition}

\begin{remark}
This makes explicit that the earlier ``kernel has traces'' view (Chapter~3) is recovered from
the relational model by packaging input-output tuples into trace witnesses.
\end{remark}

\subsubsection{The model axiom: compositional semantics (normative)}

Let $\mathfrak{P}$ be a process algebra instance with sequential operator $\seqc$ and sequential kit
$(\Compat_{\seq},\Compose_{\seq})$ (Definition~\ref{def:composition-kit}).
We now state the canonical \emph{model axiom} that makes the algebra semantic.

\begin{axiom}[M0: Sequential semantics is relational composition via the kit]
\label{ax:M0-seq}
For all kernels $K_1,K_2\in\mathcal{K}$ and all inputs $(t,u,\gamma)\in T\times U\times \Gamma$,
\[
(t,u,\gamma)\xRightarrow{K_2\seqc K_1}(u_2,v_{21},d_{21},r_{21},c_{21},\gamma_2)
\]
iff there exist intermediate outputs $(u_1,v_1,d_1,r_1,c_1,\gamma_1)$ and stage outputs
$(u_2,v_2,d_2,r_2,c_2,\gamma_2)$ such that:
\begin{enumerate}
  \item $(t,u,\gamma)\xRightarrow{K_1}(u_1,v_1,d_1,r_1,c_1,\gamma_1)$,
  \item $(t,u_1,\gamma_1)\xRightarrow{K_2}(u_2,v_2,d_2,r_2,c_2,\gamma_2)$,
  \item letting
  \[
  \tau_1=(t,u,\gamma;\ u_1,v_1,d_1,r_1,c_1,\gamma_1),\qquad
  \tau_2=(t,u_1,\gamma_1;\ u_2,v_2,d_2,r_2,c_2,\gamma_2),
  \]
  we have $\Compat_{\seq}(\tau_1,\tau_2)$,
  \item and the composite output components $(v_{21},d_{21},r_{21},c_{21})$ are exactly those prescribed by the constructor
  \[
  \tau_{21} := \Compose_{\seq}(\tau_1,\tau_2)\in T^\star.
  \]
\end{enumerate}
Equivalently: the semantics of $K_2\seqc K_1$ is the relational composition of $\Exec_{K_1}$ and $\Exec_{K_2}$
\emph{factored through} the declared trace constructor.
\end{axiom}

\begin{remark}
Axiom~\ref{ax:M0-seq} is the semantic counterpart of the ``compositional generation'' assumption used in
conformance preservation (Chapter~6). Once M0 is adopted, those preservation theorems become theorems
about the model class rather than extra assumptions.
\end{remark}

\begin{axiom}[M1: Parallel semantics via the kit (optional)]
\label{ax:M1-par}
If $\parc$ and a parallel kit $(\Compat_{\par},\Compose_{\par})$ are present, then $\Exec_{K_1\parc K_2}$
is defined by the analogous relational product/join construction using $\Compose_{\par}$.
\end{axiom}
